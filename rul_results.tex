\documentclass[12pt]{article}

% -------------------------------
% Packages
% -------------------------------
\usepackage{graphicx}
\usepackage{amsmath, amssymb}
\usepackage{caption}
\usepackage{subcaption}
\usepackage{float}
\usepackage{geometry}
\geometry{margin=1in}

% -------------------------------
% Title
% -------------------------------
\title{\textbf{Experimental Results: Forecasting Behavior, Error Analysis, and RUL Estimation}}
\author{SeYoon Oh}
\date{}

\begin{document}
\maketitle

\section*{Overview}
This appendix presents the detailed visual results used in the analysis of  
Transformer-based forecasting, reconstruction error modeling, failure-onset detection,  
and Remaining Useful Life (RUL) estimation on the IMS Bearing dataset.  
All figures shown here directly correspond to the experimental pipeline described in  
Sections 4 and 5 of the dissertation.

% ============================================================
% FIGURE 1 — Forecasting + Error Curve
% ============================================================
\begin{figure}[H]
    \centering
    \includegraphics[width=\textwidth]{figures/rul_prediction_1to2.png}
    \caption{
        \textbf{Forecasting Behavior and RUL Estimation (1st\_test $\rightarrow$ 2nd\_test).}
        The top panel compares actual vibration signals (blue) with next-step Transformer predictions (orange).
        During early operation, both curves exhibit high self-similarity, showing that the model successfully 
        learns the normal-operation manifold.  
        In later stages, spectral mismatch between the 1st and 2nd test domains yields a progressive divergence 
        in the predicted trajectory.  
        The bottom panel displays the time-evolution of the reconstruction MSE (log scale).  
        The vertical red line marks the detected failure onset, where error escalation becomes sustained and monotonic,
        confirming the reliability of the proposed detection method.
    }
    \label{fig:rul_prediction_1to2}
\end{figure}

% ============================================================
% FIGURE 2 — Error Distribution + Gaussian Fit
% ============================================================
\begin{figure}[H]
    \centering
    \includegraphics[width=0.95\textwidth]{figures/error_distribution_gaussian.png}
    \caption{
        \textbf{Reconstruction Error Distribution and Gaussian Threshold Derivation.}
        The histogram shows the empirical distribution of reconstruction errors computed only from the 
        variance-stable ``normal region.''  
        A Gaussian model (red curve) is fitted to derive the statistical parameters 
        $\mu$ and $\sigma$, which form the basis of the adaptive warning thresholds.  
        Thresholds for Low ($T$), Medium ($1.5T$), and High ($2T$) warning levels are drawn as vertical dashed lines.  
        This demonstrates that the proposed thresholding mechanism is data-driven and domain-aware,
        unlike conventional fixed-threshold IMS methods.
    }
    \label{fig:error_distribution}
\end{figure}

% ============================================================
% FIGURE 3 — Rolling Variance Spike Detection
% ============================================================
\begin{figure}[H]
    \centering
    \includegraphics[width=\textwidth]{figures/rolling_variance_spike.png}
    \caption{
        \textbf{Normal Region Detection via Rolling Variance Spike Analysis.}
        The blue curve denotes the reconstruction error sequence, while the orange curve shows the rolling variance 
        (window size = 200).  
        The red dashed line indicates the spike threshold, and the red vertical line marks the first occurrence of a 
        variance spike, which separates the statistically stable normal region from the degradation phase.  
        Threshold estimation is performed \emph{only} on the pre-spike region, preventing failure-induced variance 
        inflation from corrupting baseline statistics.  
        This mechanism is essential for stable anomaly scoring in non-stationary industrial time-series signals.
    }
    \label{fig:variance_spike}
\end{figure}

% ============================================================
% EXPLANATORY TEXT (For Dissertation Supplement)
% ============================================================

\section*{Interpretation and Discussion}

\subsection*{Forecasting Deviation and Failure Onset Behavior}
As illustrated in Fig.~\ref{fig:rul_prediction_1to2}, the Transformer effectively reconstructs the 
normal manifold during early operation.  
Cross-domain evaluation (1st\_test $\rightarrow$ 2nd\_test) introduces additional spectral mismatch, causing prediction 
deviation to accumulate gradually rather than abruptly.  
The failure onset aligns with the point where the MSE curve transitions from bounded oscillation to 
positive-drift escalation, confirming the validity of the anomaly-based onset detection rule.

\subsection*{Gaussian-Based Adaptive Thresholding}
The error distribution in Fig.~\ref{fig:error_distribution} reveals that normal-region errors are unimodal and 
near-Gaussian, allowing statistically principled thresholds to be computed.  
This resolves a major weakness of IMS literature, where fixed global thresholds often fail due to domain-dependent 
noise differences.  
The proposed adaptive thresholds scale with $\sigma$, yielding stable warning classification across Test 1/2/3.

\subsection*{Normal Region Extraction via Rolling Variance}
Fig.~\ref{fig:variance_spike} demonstrates that rolling variance is an effective signal for detecting 
the boundary between stable and unstable operation.  
Unlike heuristic ``first 10\%'' approaches, this spike-based method is fully data-driven, robust to noise,  
and applicable to cross-domain scenarios where degradation dynamics differ significantly.

\end{document}
